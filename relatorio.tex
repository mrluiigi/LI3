\documentclass[10pt]{article}

\usepackage[utf8]{inputenc}
\usepackage[portuguese]{babel}
\usepackage{a4wide}
\usepackage{listings}


\title{Projeto de Laboratórios de Informática 3\\Grupo 26}
\author{José Pinto (A81317) \and Luís Correia (A81141) \and Pedro Barbosa (A82068)}
\date{\today}

\begin{document}

\maketitle

\begin{abstract}
  Neste projeto da disciplina de Laboratórios de Informática 3 (LI3) do Mestrado Integrado em Engenharia Informática da Universidade do Minho foi-nos proposto desenvolver um sistema em C capaz de processar ficheiros XML que armazenam informações utilizadas pelo Stack Overflow, podendo depois executar um conjunto de querys de forma eficiente.
\end{abstract}

\tableofcontents

\section{Introdução}
\label{sec:intro}

Neste relatório vamos primeiramente falar das estruturas de dados utilizadas e dos vários módulos criados, terminando com a explicação das estratégias usadas em cada query.  

\section{Dependências externas}
\label{sec:dependencias}

Este projeto tem duas dependências externas, as bibliotecas libxml2 e glib.

\section{Módulos}
\label{sec:modulos}

Nesta secção vamos falar dos módulos criados para auxiliar o desenvolvimento deste projeto.

\subsection{Post}

	Este módulo corresponde à implementação de um post do Stack Overflow.
\subsubsection{Estrutura de dados}
	A informacão relativa a um utilizador é guardada na estrutura \emph{post}:
	\begin{lstlisting}
	struct post {
		//Identifica o tipo de post (Pergunta ou Resposta)
		char postTypeId;
		QUESTION q; 
		ANSWER a; 
		//ID do post 
		int id;
		//ID do utilizador que fez o post 
		char * ownerUserId;
		//Data da criacao do post 
		Date creationDate;
	};
	\end{lstlisting}
	Esta estrutura é a implementação concreta do tipo de dados abstrato POST declarado no header:
	\begin{lstlisting} 
	typedef struct post * POST;
	\end{lstlisting}

	Como a informação de um post varia conforme seja uma pergunta ou uma resposta, foram criadas as estrututras QUESTION e ANSWER:
\begin{lstlisting}
	typedef struct question{
		//Titulo
		char *title;
		//Numero de respostas 
		int nanswers;
		//Tags 
		GSList *tags;
		//Ultima atividade do post 
		Date lastActivityDate;
	}*QUESTION;
	\end{lstlisting}
	\begin{lstlisting}
	typedef struct answer{
		// ParentID (ID do post a que a resposta pertence) 
		int parentId;
		// Numero de comentarios de uma resposta 
		int comments;
		// Score de uma resposta 
		int score;
	}*ANSWER;
	\end{lstlisting}

	O membro \emph{tags} é uma lista ligada que guarda os ids das tags de um post.

\subsubsection{Funções}

	Devido à dicotomia dos posts, existem duas funções distintas para a criação de uma instância de um POST. Estas criam cópias de todos parâmetros recebidos com tipo não primitivo, para garantir o encapsulamento dos dados.

	O acesso exterior aos membros da estrutura é realizado exclusivamente através de funções. Todas as variáveis podem ser consultadas, sendo o resultado da consulta de tipos não primitivos uma cópia destes. Nenhum membro da estrutura pode ser alterado após a iniciaização.

	Existe também uma função que liberta a memória alocada para a estrutura.

\subsection{Posts}
	O objetivo deste módulo é permitir guardar e processar a informação de todos os posts de uma comunidade do Stack Overflow
\subsubsection{Estrutura de dados}

	A estrutura utilizada é \emph{TCD\_posts}:
	\begin{lstlisting}
	struct TCD_posts {
		GSList *list;
		GHashTable *hash;
		GHashTable *months_hash; 
};
	\end{lstlisting}

	O membro \emph{list} é uma lista ligada que contêm todos os posts por ordem cronológica inversa. \emph{hash} é uma hashtable que associa o id de um \emph{POST} à posicão deste na lista. \emph{months\_hash} é uma hashtable que faz a associação entre um mês(e o ano) e o primeiro post desse mês. Assim, quando um dos parâmetros de uma interrogação é um intervalo de tempo, é possível começar a percorrer a lista dos posts num ponto já muito próximo do intervalo pedido, evitando percorrer a lista desnecessariamente. Desta forma o tempo de execução dessas interrogações é muito menor. Poderíamos ter optador por guardar o primeiro post de cada dia no entanto a melhoria no tempo de execução não compensaria o tamanho acrescido da hashtable. 

	Esta estrutura é a implementação concreta do tipo de dados abstrato \emph{Posts} declarado no header:
	\begin{lstlisting} 
	typedef struct TCD_posts * Posts;
	\end{lstlisting}

	Como é necessário percorrer a lista dos posts em várias interrogações também foi definido o tipo \emph{PostsList} para manter a abstração:
	\begin{lstlisting} 
	typedef GSList * PostsList;
	\end{lstlisting}

\subsubsection{Funções}	
	
	Depois de uma instância de \emph{Posts} ser inicializada com \emph{init\_posts}, podem ser utilizadas funções para adicionar/remover posts. Tal como no módulo anterior existe uma função de adição para as perguntas e outras para as respostas. Estas funções recebem a informação através de vários parâmetros e usam internamente as funções do módulo Post para assegurar o encapsulamento.

	Para garantir o correto funcionamento do módulo é necessário que o a função \emph{finalize\_posts} seja invocada após todos os posts serem adicionados. O bom funcionamento do módulo estar dependente de um fator externo pode não ser ideal mas foi um compromisso realizado para garantir um tempo de execução aceitável. É mais eficiente ordenar a \emph{list} e preencher a \emph{months\_hash} depois de todos os posts serem inseridos.

	Existem várias funções que permitem procurar um post, ou a localização deste na lista, de acordo com certos parâmetros (id, data, etc). Também existem funções que permitem percorrer a lista de posts e obter o \emph{POST} dessa posição. Como estas últimas funções recebem/devolvem o tipo abstrato \emph{PostsList} o encapsulamento é garantido.

	Também está definida uma função para libertar a memória alocada.

\subsection{MyUser}
	Este módulo corresponde à implementação de um utilizador do Stack Overflow, contendo a informação relativa a um \emph{user}. 
\subsubsection{Estrutura de dados}
	
	A informacão relativa a um utilizador é guardada na estrutura \emph{user\_ht}:
	\begin{lstlisting}
	struct user_ht{
		//ID do user 
		int id;
		//Nome do utilizador 
		char *name;
		// ShortBio do utilizador 
		char *shortBio;
		// Numero de posts do utilizador 
		unsigned short nr_posts;
		// Ultimo post do utilizador 
		int lastPost;
		// Reputacao do utilizador 
		int reputation;
	};
	\end{lstlisting}


	O último post do utilizador corresponde ao id do post mais recente deste e a sua utilidade será esclarecida na secção sobre a interrogação 5. Os restantes membros da estrutura contêm informação necessária para responder às interrogações.

	Esta estrutura é a implementação concreta do tipo de dados abstrato MY\_USER declarado no header:
	\begin{lstlisting} 
	typedef struct user_ht * MY_USER;
	\end{lstlisting}

\subsubsection{Funções}

	A criação de uma instância do MY\_USER é feita através de um método de construção parametrizado. Esta função tem o cuidado de criar cópias das strings recebidas para garantir o encapsulmanento dos dados.

	O acesso exterior aos membros da estrutura é realizado exclusivamente através de funções. Todos as variáveis podem ser consultadas, sendo o resultado da consulta de tipos não primitivos uma cópia destes. As únicas que podem ser modificadas após a inicialização são o \emph{lastPost} e o \emph{nr\_posts} devido ao facto de que a informação destes não se encontrar no ficheiro Users.xml e ser obtida através do posterior processamento do ficheiro Posts.xml

	Por último, existe uma função que liberta a memória alocada para a estrutura.

\subsection{Users}
	O objetivo deste módulo é permitir guardar e processar a informação de todos os utilizadores de uma comunidade do Stack Overflow
\subsubsection{Estrutura de dados}

	A estrutura utilizada é \emph{users}:
	\begin{lstlisting}
	struct users {
		GHashTable* hash;
		GSList* users_by_nr_posts;
		GSList* users_by_reputation;
	};
	\end{lstlisting}

	\emph{hash} é uma hashtable que associa o id a um \emph{USER}. \emph{users\_by\_nr\_posts} e \emph{users\_by\_nr\_reputation} são listas ligadas de \emph{USER}, ordenadas em ordem decrescente de acordo com o número de posts e a reputação de cada \emph{USER} respetivamente.

	Esta estrutura é a implementação concreta do tipo de dados abstrato MY\_USER declarado no header:
	\begin{lstlisting} 
	typedef struct users * USERS;
	\end{lstlisting}

\subsubsection{Funções}
	Tal como no módulo \emph{Posts}, depois de uma instância de \emph{Users} ser inicializada com \emph{init\_users}, podem ser utilizadas funções para adicionar/remover users. A função de adição recebe a informação através de vários parâmetros e usam internamente as funções do módulo User para assegurar o encapsulamento.

	Existem duas funções para alterar o número de posts e o último post de um dado utlilizador. Assim, é possível percorrer a lista de posts contida numa instância de \emph{Posts} para alterar estas variáveis mesmo após as instâncias de \emph{USER} já terem sido criadas.

	Por razões semelhantes às explicadas no módulo \emph{Posts} (neste caso é a ordenação das duas listas), é necessário que o a função \emph{finalize\_users} seja invocada após todos os utilizadores serem adicionados/alterados.

	A funcionalidade principal deste módulo é a função \emph{find\_user}, que permite obter a informação de um utilizador dado o seu id, e duas funções que percorrem uma das listas da estrura \emph{users} e criam e retornam uma \emph{LONG\_list} com os N utilizadores com mais posts ou mais reputação.

	Também está definida uma função para libertar a memória alocada.

\subsection{My Date}

	O módulo \emph{mydate} é um pequeno módulo que possui algumas funções relacionadas com datas que são utlilizadas por vários módulos. 

\subsection{Tags}
	Este módulo é responsável por guardar e processar informação sobre tags de posts
\subsubsection{Estrutura de dados}
	A estrutura de dados deste módulo é apenas uma hash table que associa o nome de uma tag ao seu id. Para o propóśito da abstração esta tabela é representada pelo tipo \emph{TAGS}:
	\begin{lstlisting} 
	typedef GHashTable * TAGS;
	\end{lstlisting}

\subsubsection{Funções}
	As únicas funções presentes neste módulo são as de inicialização, inserção, procura e libertação de memória. Existe apenas mais uma função utilizada para converter a informação sobre tags presente no Posts.xml numa lista para passar à função \emph{create\_question} do módulo \emph{Post}

\section{Interrogações}

	Nesta secção vamos explicar os métodos usados para cada uma das interrogações e as estratégias para melhoramento de desempenho.

\subsection{Interrogação 1}
	Nesta interrogação usa-se primeiramente uma função para devolver o post com o ID dado. Caso o post seja uma pergunta, procura-se o user que fez esse post para que se devolva o nome do autor junto com o título do post. Se o post for uma resposta, devolve-se o nome do user que fez a resposta e procura-se pela pergunta a que essa resposta diz respeito para se retornar também o título da pergunta. 

\subsection{Interrogação 2}

	A struct dos users, formada por duas listas e uma hashtable, já possui uma lista com todos os users ordenados pelo número de posts, pelo que apenas é necessária uma função que percorra essa mesma lista e devolva uma nova com os N users pretendidos.

\subsection{Interrogação 3}

	No módulo post\_list temos a lista de posts ordenada da mais recente para a mais antiga.
	Procuramos com a ajuda da hashtable months\_hash o post anterior ao intervalo de tempo dado, retornando a lista de posts a partir desse post. Com isto, percorre-se a lista dentro do intervalo de tempo contabilizando o número de perguntas e respostas.

\subsection{Interrogação 4}

	Inicialmente utilizamos a hashtable months\_hash para encontrar o último post que ainda faz parte do intervalo de tempo em questão. De seguida, percorremos a lista de posts desde esse mesmo post até chegar à data do begin. À medida que se percorre cada post verifica-se se é uma pergunta e se contém a tag. Se estas condições se verificarem adicionamos o ID do post a uma lista ligada. Por fim, copiam-se todos os IDs da lista ligada para uma LONG\_list e retornamos a mesma.

\subsection{Interrogação 5}

	A partir do ID dado, procura-se na hashtable dos users pelo user com esse ID. Com isto obtém-se a informação do seu perfil (short bio), o número de posts e o último post do user.
	Começa-se a percorrer a lista dos posts desde o último post até serem encontrados 10 ou já se tiver percorrido todos os posts do user.
	Cada post pertencente ao user é adicionado a um array
	Caso não sejam encontrados 10 posts, o resto do array fica a 0.

\subsection{Interrogação 6}

	Inicialmente utilizamos a hashtable months\_hash para encontrar o último post que ainda faz parte do intervalo de tempo em questão. De seguida, percorremos a lista de posts desde esse mesmo post até chegar à data do begin. Para cada post, caso o post seja uma resposta é adicionado a uma lista ligada temporária. Após percorrer todos os posts, ordenamos a lista temporária segundo a score de cada resposta. Por fim, copiamos as primeiras N respostas para uma LONG\_list e retornamos a mesma. Caso não sejam retornadas N respostas no intervalo dado, preenchemos o resto da lista com zeros.

\subsection{Interrogação 7}

	No módulo post\_list temos a lista de posts ordenada da mais recente para a mais antiga.
	Procuramos com a ajuda da hashtable months\_hash o post anterior ao intervalo de tempo dado, retornando a lista de posts a partir desse post.
	Percorre-se a lista de posts dentro do intervalo de tempo dado e caso um post seja uma pergunta, adiciona-se a uma lista ligada temporária.	
	No fim de se percorrer a lista de posts, ordena-se a lista ligada temporária de acordo com o número de respostas, para que se possa retornar os IDs das N perguntas com mais respostas.
	Se não foram encontrados N perguntas no intervalo de tempo dado, o resto da LONG\_list fica a 0.

\subsection{Interrogação 8}

	Percorrer a lista dos posts e verificar se é uma pergunta e se a palavra pertence ao seu título, caso pertença adiciona-se o ID da pergunta a uma LONG\_list e devolve-se a mesma. Caso não sejam retornadas N perguntas com a palavra no título, preenchemos o resto da lista com zeros.

\subsection{Interrogação 9}

	Para resolver a query criamos três queues, na queue1 vamos guardar respostas do utilizador com ID1 e na queue2 vamos guardar respostas do utilizador com ID2 e na queuef guardamos o ID dos posts onde participam ambos os utilizadores. À medida que percorremos a lista dos posts, quando se encontra uma resposta verificamos quem a publicou, se for o ID1/2 verificamos o parentOwnerId, se este for o ID2/1 adicionamos o ID da pergunta à queuef, se o parentOwnerId não for o ID2/1, vamos à queue2/1 verificar se tem alguma respostas com o mesmo parentOwnerId, se tiver adicionamos à queuef. Caso nenhuma das condições se verifique adicionamos à queue1 ou à queue2. Quando o post que encontramos é uma pergunta se o parentOwner for o ID1/2, vamos à queuef verificar se o post está lá, se estiver adicionamos à LONG\_list, caso contrário não faz nada.

\subsection{Interrogação 10}

	Primeiramente, utiliza-se a hashtable hash da estrutura dos posts, para obter a lastactivityDate do post. De seguida, utiliza-se a hashtable months\_hash para encontrar o primeiro post que antecede o post cuja data é a lastActivityDate. Por fim percorremos a lista até à creationDate da pergunta, verificando se o post encontrado é uma resposta à pergunta. Por fim é calculado a score de cada post e comparam-se todas as scores até encontrar a melhor retornado o ID dessa resposta.

\subsection{Interrogação 11}

	Inicialmente, fazemos uma lista com os N utilizadores com melhor reputação. Seguidamente utilizamos a hashtable months\_hash para encontrar o último post que ainda faz parte do intervalo de tempo em questão e percorremos a lista de posts desde esse mesmo post até chegar à data do begin. À medida que se percorre a lista dos posts, caso seja uma pergunta e o owner seja um dos N utilizadores com melhor reputação vamos adicionar a tag a uma lista. Caso a tag já esteja presente na mesma, incrementamos o número de vezes que ocorre. No final, ordenamos a lista das tags, copiamos para uma LONG\_list e retiramos as N primeiras. Caso não sejam contadas N tags, enchemos o resto da lista com zeros.

\end{document}