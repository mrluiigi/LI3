\documentclass[a4paper]{article}

\usepackage[utf8]{inputenc}
\usepackage[portuges]{babel}
\usepackage{a4wide}

\title{Projeto de Laboratórios de Informática 3\\Grupo 26}
\author{José Pinto (A81317) \and Luís Correia (A81141) \and Pedro Barbosa (A82068)}
\date{\today}

\begin{document}

\maketitle

\begin{abstract}
  Neste projeto da disciplina de Laboratórios de Informática 3 (LI3) do Mestrado Integrado em Engenharia Informática da Universidade do Minho foi-nos proposto desenvolver um sistema em C capaz de processar ficheiros XML que armazenam informações utilizadas pelo Stack Overflow, podendo depois executar um conjunto de querys de forma eficiente.
\end{abstract}

\tableofcontents

\section{Introdução}
\label{sec:intro}

Neste relatório vamos primeiramente falar das estruturas de dados utilizadas e dos vários módulos criados, terminando com a explicação das estratégias usadas em cada query.  

\section{Módulos}
\label{sec:modulos}

Nesta secção vamos falar dos módulos criados para 

\subsection{Post}

	Neste módulo é guardada toda a informação relativa a um post, para isso foi implementada a estrutura POST onde guardamos o tipo do post, se é pergunta ou resposta, o id do post, o id do utilizador que fez o post e a data da criação do post. Para além disso, se o post for uma pergunta, a estrutura do post tem também a informação relativa à pergunta que é guardada na estrutura QUESTION. Se o post for uma resposta o POST tem a informação guardada na estrutura ANSWER.
	
	Na estrutura QUESTION é guardado o título da pergunta, o número de respostas a essa pergunta, as tags e a data da última atividade do post.
	
	A estrutura ANSWER possui o parentId que é o id do post a que a resposta pertence, o número de comentários e o score dessa resposta.

	Este módulo possui dois métodos construtores parametrizados, um para a QUESTION e outro para a ANWSER. Também possui métodos \emph{get} que retornam cópias de conteúdo para que se garanta o encapsulamento e funções que verificam se um determinado post é uma pergunta ou resposta e se contém uma determinada tag.

	Por último, existe uma função que liberta a memória alocada para um post.

\subsection{Posts list}
	
	Este módulo guarda a informação relativa aos \emph{posts} numa struct, \emph{TCD Posts}, formada por uma lista ligada e duas hashtables. Dentro da lista é feita a distinção entre os dois tipos de posts que podemos ter, pergunta ou resposta, e as hashtables são utilizadas para percorrer mais rapidamente a lista dos posts, sendo que uma delas guarda o primeiro post de cada mês para facilitar a execução de uma das querys.

	Primeiramente, inicializamos a struct com uma função \emph{init}.

	Para adicionar um post à struct utilizamos duas funções distintas, uma para as perguntas, \emph{add question to posts}, e uma para as respostas, \emph{add answer to posts}.

	Ao longo deste módulo são definidos vários tipos de \emph{find}, um para devolver todos os posts a seguir a uma data passada como argumento, \emph{find by date}, um que encontra um posts específico e retorna o mesmo, \emph{find post}, um que retorna todos os posts a seguir ao encontrado, \emph{find post in list} e por fim uma função que dadas duas listas retorna aquela que contém o post mais recente, \emph{find most recent post}.

	Para o caso em que um post é uma resposta, foi definida a função, \emph{get parent owner}, que devolve o ID do utilizador que publicou a pergunta à qual o post em questão responde. Para além deste \emph{get} foram definidos outros três, \emph{get posts list}, \emph{get post} e \emph{get next}.

	Para terminar, foram ainda definidas duas funções de comapração, \emph{compare nanswers} e \emph{compare date list} cujo objetivo é identificar o post com mais respostas e comparar a data do primeiro post de duas listas diferentes respetivamente. Também estão definidas duas funções \emph{free} para libertar a memória alocada para as structs.

\subsection{My User}

	Este módulo guarda a informação relativa a um \emph{user}. Existe a estrutura \emph{user ht} que guarda o id, o nome, o shortBio, o número de posts, o último post desse utilizador e a sua reputação.

	Foi adicionado um método de construção parametrizado e métodos \emph{get} que retornam cópias de conteúdo para que se garanta o encapsulamento.

	Por último, existe uma função que liberta a memória alocada para um post.

\subsection{Users}

	O presente módulo permite guardar toda a informação relativa a todos os \emph{users}. Para isso utiliza-se uma estrutura que contém uma \emph{hashtable} e duas listas ligadas, uma delas ordenada conforme o número de posts e a outra segundo a reputação de cada \emph{user}. A \emph{hashtable} tem como objetivo encotnrar um post dado o ID do utilizador.

	Para tornar possivel a organização das duas listas ligadas guardadas, é necessário definir uma série de funções. Inicialmente utilizamos uma função para inicializar ambas as listas e a \emph{hashtable}. para adicionar e remover utilizadores da lista utilizamos as funções \emph{add myuser} e \emph{remove myuser}. No final utiliza-se uma função \emph{free} para libertar a memória alocada para a estrutura.

	De maneira a conseguir contar o número total de posts de um utilizador utilizamos a função \emph{find and increment user nr posts} que procura por um utilizador e aumenta o seu número de posts em uma unidade.

	De forma tornar possivel a organização das listas conforme os parâmetros desejados implementamos duas funções, uma que compara os utilizadores conforme a sua reputação ou número de posts, \emph{compare users by nr posts} e \emph(compare users by reputation), e duas funções sort, \emph{sort users by nr posts} e \emph{sort users by reputation}, que utilizam a função da biblioteca GLIB e as auxiliares definidas anteriormente para organizar a lista segundo o parâmetro desejado. No final, para garantir que os users estão de facto ordenados segundo o parâmetro desejado, utilizamos a função \emph{finalize users}.

	Neste módulo estão ainda definidas algumas funções adicionais que sãao utilizadas como auxiliares de maneira a tornar possivel aa execução correta de algumas querys tais como a \emph{get N users with most reputation} e \emph{get N users with most nr posts} que tal como o nome indica, devolvem os N utilizadores com melhor reputação ou com o maior número de posts, a função \emph{find user} que encontra um user dado o seu ID e a \emph{find and set user lastPost} que dado um utilizador o encontra na lista e define o último post publicado pelo mesmo.


\subsection{My Date}

	O módulo \emph{mydate} é um módulo simples que compara duas datas, \emph{compare date}, converte a data escrita no XML numa \emph{date}, \emph{xmlCreationDate to Date}, e uma função que converte a data do XML numa key para ser usada na monthshash da struct dos posts, \emph{date to Key}.


\subsection{Tags}

	A struct utilizada para guardar as tags é uma Hashtable onde o ID de cada posição é o ID da tag e o conteúdo é a própria tag. Para inserir uma tag na hashtable utilizamos a função \emph{insert tags}. Neste módulo definimos uma função \emph{getTags} que dado uma string com as várias tags utilizadas num post as converte numa lista ligada. Temos também uma função que dada uma tag devolve o seu ID, \emph{convert tag naame to id}. Como é costume, está definida uma função free que liberta a memória alocada para a struct.


\end{document}